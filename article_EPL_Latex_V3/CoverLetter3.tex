\documentclass[page-classic]{epl2}


\begin{document}

Dear Editors,\\

Thanks for the review of our article. Please see attached the
revised version of the paper, modified in accordance to referees
recommendations. The changes are highlighted using revision command, as recommended.
See below the list of changes and replies to referee comments.

With best regards, \\
Alexei Buzulutskov \\
Corresponding author

\vspace{0.5 cm}

 List of changes and replies to the comments.

\vspace{0.5 cm}


REPORT A

\vspace{0.5 cm}

REFEREE A - Second Report

\vspace{0.5 cm}

Comment 1. I will ignore the ramblings of the author regarding
lobbies etc in their reply and concentrate on their manuscript.  I
acknowledge their additions to clarify the gamma background, which
they themselves agreed was a weakness. This has been resolved to a
reasonable extent. It is not clear why a full Monte Carlo of the
actual experiment could not be performed, but this at least sets the
scale for the gamma contamination that could otherwise be perceived
to bias their result.

\vspace{0.5 cm}

Reply: thanks.

\vspace{0.5 cm}

Major comments

Comment 2. I maintain my comments regarding the introduction. The
suggestion that the discrepancy between the null result  experiments
and those who claimed to observe a signal cannot be explained by any
reasonable systematic uncertainty associated with the calibration of
the noble liquid detectors. In any case the authors go on to
recognize that that there is ample experimental data on such yields
[for LXe]. So are they suggesting that all those measurements are
wrong?

My comment on the relevance of this measurement for the detection of
coherent neutrino scattering was ignored.  None of these things take
merit away from the work, as I tried to explain in my previous
review. Rather, its value is to elucidate on a microscopic
understanding of the ionization yield in liquid argon which
complements other measurements conducted at lower energies. Those,
e.g. that from Joshi et al, are indeed relevant for coherent
neutrino scattering. I urge the authors to reconsider those
introductory paragraphs if they want their paper to be taken
seriously.

\vspace{0.5 cm}

Reply. We finally agreed with this comment. The introduction has
been accordingly modified: namely the motivation based on confusion
situation in dark matter search, as well as two references trying to
explain that by the calibration problem, have been removed. Coherent
scattering is not anymore used as a basic motivation for current
study. Moreover, we mention that our results complement those at
lower energies. Now the first two paragraphs of the introduction are
as follows.

\vspace{0.5 cm}

The energy calibration of nuclear recoil detectors is of
\revision{primary} importance to rare-event experiments
\cite{NobleRev} such as those of \revision{direct} dark matter
search
\cite{Xenon10,Xenon100,Zeplin3,Warp,Lux,ArDM,Darkside,Dama,Cogent,Crest,Cdms}
and coherent neutrino-nucleus scattering \cite{CoNu1,CoNu2}. Such a
calibration, in particular in liquid Ar and Xe detection media, is
usually performed by measuring the ionization yield and
scintillation efficiency of nuclear recoils, using neutron elastic
scattering off nuclei (the latter imitating the interaction with
\revision{dark matter particle} or coherently scattered neutrino).
While for liquid Xe there is an ample of experimental data on such
yields \cite{LXeYield1,LXeYield2,LXeYield3}, little is known  about
the ionization yield \cite{Joshi,Cao} and scintillation efficiency
\cite{LArScint} in liquid Ar.

Recently the first results on the ionization yield of nuclear
recoils in liquid Ar have been presented, in the lower energy range:
at 6.7 keV \cite{Joshi} and 17-57 keV \cite{Cao}. In the present
work, the ionization yield of nuclear recoils in liquid Ar has for
the first time been measured at higher energies, at 80 and 233 keV.
\revision{These results complement those measurements conducted at
lower energies and thus might be relevant to the future dark matter
search experiments \cite{ArDM,Darkside} and to thorough
understanding of the ionization yield in liquid Ar.} ...

\vspace{0.5 cm}

Comment 2. My only remaining major comment is that regarding the
energy dependence of the yield:  although I agree that it is worth
pointing out that it appears similar to that predicted by the Jaffe
model, it is not reasonable not to point out (in the abstract) that
there is a significant difference in absolute value. In the
conclusions this disagreement is described as slight, which is
inappropriate. Figure 6 shows this to be of the order of 60\% - this
is not slight, it is significant. I propose that they amend their
new sentence in the abstract to include this: The energy dependence
of + is similar to that predicted by the Jaffe model, although the
absolute value is substantially lower or words to that effect. This
is, after all, a significant result of their study.

\vspace{0.5 cm}

Reply. In fact the Jaffe model perfectly describes even the absolute
values of the ionization yield if the ratio Nex/Ni is taken as a
free parameter, resulting in Nex/Ni=2.3. Accordingly we don't need
to modify the conclusions and abstract. However, such a large ratio
value needs the justification. To clarify the situation with
Jaffe model, we changed the Fig.6 adding to it the Jaffe model curve
at Nex/Ni=2 and modified the last paragraph in theoretical model
section as follows.

\vspace{0.5 cm}

\revision{In the lack of theoretical and experimental data for
liquid Ar, the ratio $N_{ex}/N_i$ can be taken here either equal to
that measured for nuclear recoils in liquid Xe \cite{Sorensen11},
namely $N_{ex}/N_i$=1, or 10 times greater than the ratio for
electron recoils in liquid Ar (in the same way as it was for Xe -
see table 2 in \cite{NobleRev}), namely $N_{ex}/N_i$=2. One can see
that the Jaffe model can probably consistently describe the
experimental data in terms of the energy dependence, and even in
terms of the absolute values in the latter case.}

\vspace{0.5 cm}

Minor comments

\vspace{0.5 cm}

Comment 3. Line 43: parameterization rather than parameterizations

\vspace{0.5 cm}

Reply: corrected.

\vspace{0.5 cm}

Comment 4 Line 46. The terminology electron-equivalent recoils is
misleading here.  The authors are probably referring to electron
recoils, the calibration of which is often characterized by an
electron equivalent energy.

\vspace{0.5 cm}

Reply: agree. Corrected as follows.

\vspace{0.5 cm}

Equations 1 and 2 are valid for both \revision{electron recoils},
induced by electron or gamma-ray irradiation, and nuclear recoils;
\revision{it is conventional to refer to the corresponding recoil
energy in units of keVee (electron-equivalent) and keVnr}.

\vspace{0.5 cm}

Comment 4. Experimental setup. New paragraph added, starting line
106:  it is not at all obvious how the neutron detector will later
on be used to determine the photon flux in the LAr chamber; a short
sentence would be useful here.

\vspace{0.5 cm}

Reply: agree. The paragraph is modified as follows.

\vspace{0.5 cm}

\revision{In addition, a neutron scintillation counter made of
stilbene (C$_{14}$H$_{12}$) was enabled \cite{NCount}; it was placed
close to the CRAD active volume, just underneath the neutron
generator. The counter could effectively separate neutrons from
gammas using a pulse-shape analysis and thus estimate the gamma-ray
background due to ($n,\gamma$) reactions in the two-phase CRAD.}


\vspace{0.5 cm}

Comment 5. Experimental results. Line 121: using THE 59.5 keV line (missing word)

\vspace{0.5 cm}

Reply: corrected

\vspace{0.5 cm}

Comment 6. Line 127: suggest indicating that the resolution is
expected to be reasonably spatially uniform, since this is relevant
later on for the neutron data.

\vspace{0.5 cm}

Reply: agree. Corrected as follows.

\vspace{0.5 cm}

The resolution \revision{is expected to be reasonably spatially
uniform and} practically independent of the energy ...

\vspace{0.5 cm}

Comment 7. Line 153: for THE given x-ray tube (missing word)

\vspace{0.5 cm}

Reply: corrected.

\vspace{0.5 cm}

Comment 8. Line 181: by THE Klein-Nishina formula (missing word)

\vspace{0.5 cm}

Reply: see reply to the next comment.

\vspace{0.5 cm}

Comment 9. Line 182: which can be approximated by a decreasing
function of energy:  this suggests that the KN formula can itself be
approximated by such a function, but that statement refers I believe
to the differential spectrum instead: so the word which on line 182
should probably be and. In addition, the shape of the spectrum
depends on the energy deposit range and on the photon energy. This
should be clarified.

\vspace{0.5 cm}

Reply. The linear approximation is actually follows from Fig. 10.1
of ref. 37, which in turn can be derived from the KN formula. To
clarify this, the sentence is simplified, removing reference to the
KN formula, as follows.

\vspace{0.5 cm}

\revision{At these photon energies the spectrum of electron recoils
due to Compton scattering at the given recoil energies can be
approximated by a linear decreasing function with energy (see Fig.
10.1 in ref. \cite{Knoll}).}


\vspace{0.5 cm}

Comment 10. Fig 3 caption: the contribution of the latter being set
to: is set the right word here?  Also, in the last line of the
caption, the comma is unnecessary.

\vspace{0.5 cm}

Reply: corrected.

\vspace{0.5 cm}

Comment 11. Comparison with other experiments. Line 279: The models
referred to in [42] are not GEANT4 models  (i.e. they are not the
responsibility of the GEANT4 Collaboration and are not released with
GEANT4; the authors probably mean eGEANT4-based models or similar.
In any case the Monte Carlo tool is not important here.

\vspace{0.5 cm}

Reply: agree. The reference to Geant 4 has been removed:

\vspace{0.5 cm}

...predicted in some \revision{computer} simulation models
\cite{NEST}...

\vspace{0.5 cm}

Comment 12. Line 325: but slightly disagrees in terms of absolute
values is not a tenable statement as mentioned above.  There is no
way in which the disagreement shown in Fig. 6 can be described as
slight!

\vspace{0.5 cm}

Reply. See reply to comment 2.

\vspace{0.5 cm}




\end{document}
